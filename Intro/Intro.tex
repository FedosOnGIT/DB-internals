\documentclass{article}

\usepackage[utf8]{inputenc}
\usepackage[russian]{babel} 
\usepackage{amsmath} 
\usepackage{hyperref}
\usepackage{graphicx}
\usepackage{misccorr}
\usepackage{listings}
\usepackage{xcolor}

\title{DB internals. Вводная лекция}
\author{Надуткин Федор }
\date{December 2023}

\begin{document}

\maketitle

\newpage

\section*{Что будет в курсе:}

\begin{enumerate}
    \item \textbf{Composable Databases}.
    Разделение баз данных на куски, которые разрабатываются отдельно.
    
    \includegraphics*[width=0.7\textwidth]{Pictures/Разделение баз данных}

    \item \textbf{Виды СУБД}.
    OLTP (Транзакционная модель, много маленьких операций) и
    OLAP (Аналитическая модель, несколько больших операций).

    \item Оптимизация запросов.
    \begin{enumerate}
        \item Реляционное представление.
        \item Rule-based оптимизация запросов.
        \item Cost-based оптимизация: планирование Join, алгоритм Cascades.
        \item Особенности планирования в аналитических системах.
        \item Особенности планирование в распределённых системах.
    \end{enumerate}

    \item Выполнение запросов
    \begin{enumerate}
        \item Модели выполнения: pull mode, push model.
        \item Реализация операторов.
        \item Массивно-параллельное выполнение запросов.
        \item Координация выполнения запросов в распределённых системах.
        \item Интерпретируемое и компилируемое выполнение запросов.
        \item Колоночные движки и форматы.
        \item Ключевые оптимизации в OLAP.
    \end{enumerate}
\end{enumerate}

\newpage

\section*{Чего не будет}

\begin{enumerate}
    \item Транзакционная обработка(MVCC, блокироки, изоляции).
    \item Распределённые транзакции и консенсус.
    \item Отказоустойчивость.
    \item Нестандартные окружения (GPU, облака)
\end{enumerate}

\end{document}